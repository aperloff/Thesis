%!TEX root = ../TAMUTemplate.tex
%%%%%%%%%%%%%%%%%%%%%%%%%%%%%%%%%%%%%%%%%%%%%%%%%%%
%
%  New template code for TAMU Theses and Dissertations starting Fall 2016.  
%
%  Author: Sean Zachary Roberson
%	 Version 3.16.09
%  Last updated 9/12/2016
%
%%%%%%%%%%%%%%%%%%%%%%%%%%%%%%%%%%%%%%%%%%%%%%%%%%%
%%%%%%%%%%%%%%%%%%%%%%%%%%%%%%%%%%%%%%%%%%%%%%%%%%%%%%%%%%%%%%%%%%%%%
%%                           ABSTRACT 
%%%%%%%%%%%%%%%%%%%%%%%%%%%%%%%%%%%%%%%%%%%%%%%%%%%%%%%%%%%%%%%%%%%%%

\chapter*{ABSTRACT}
\addcontentsline{toc}{chapter}{ABSTRACT} % Needs to be set to part, so the TOC doesnt add 'CHAPTER ' prefix in the TOC.

\pagestyle{plain} % No headers, just page numbers
\pagenumbering{roman} % Roman numerals
\setcounter{page}{2}

\indent The Higgs boson discovery was announced on July 4th, 2012. Since then the 125\unit{\GeV} boson has been seen in many decay paths, including $H{\rightarrow}\gamma\gamma$, $H{\rightarrow}ZZ$, $H{\rightarrow}\tau\tau$, and even some $H{\rightarrow}W^{+}W^{-}$ channels.
However, no one has looked for the boson at this mass in the $H{\rightarrow}W^{+}W^{-}{\rightarrow}l{\nu}jj$.
This dissertation presents a search for the 125\unit{\GeV} Higgs in semi-leptonic W decays using both traditional kinetically discriminating variables as well as a matrix element technique.
The data for this analysis was collected in 2012 by the Compact Muon Solenoid (CMS) experiment at the Large Hadron Collider (LHC) and amounts to 19.7\unit{\fbinv} of proton-proton collisions at a center of mass energy of 8\unit{\TeV}.
Although this analysis presents a step forward in complexity, we were still not able to see a significant excess above the standard model background prediction.
However, we were able to set limits on the $\sigma\times{BR}$ for the semi-leptonic W decay of the Higgs boson.
PUT LIMITS HERE.
These represent some of the first such limits recorded.


 

\pagebreak{}


%no bold
%no references or citations
% purpose
% methods
% results
% conclusions

%Contributors and Funding sources required
%dedication and acknowledgments and nomenclature are optional
