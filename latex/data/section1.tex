%!TEX root = ../TAMUTemplate.tex
%%%%%%%%%%%%%%%%%%%%%%%%%%%%%%%%%%%%%%%%%%%%%%%%%%%
%
%  New template code for TAMU Theses and Dissertations starting Fall 2016.
%
%  Author: Sean Zachary Roberson 
%	 Version 3.08.16
%  Last updated 8/19/2016
%
%%%%%%%%%%%%%%%%%%%%%%%%%%%%%%%%%%%%%%%%%%%%%%%%%%%

%%%%%%%%%%%%%%%%%%%%%%%%%%%%%%%%%%%%%%%%%%%%%%%%%%%%%%%%%%%%%%%%%%%%%%
%%                           SECTION I
%%%%%%%%%%%%%%%%%%%%%%%%%%%%%%%%%%%%%%%%%%%%%%%%%%%%%%%%%%%%%%%%%%%%%


\pagestyle{plain} % No headers, just page numbers
\pagenumbering{arabic} % Arabic numerals
\setcounter{page}{1}


\chapter{\texorpdfstring{\uppercase {Introduction}}{Introduction}}
\begin{comment}
1) Physics/colliders
2) Standard model
3) Higgs
4) This dissertation toppic
5) Organization
\end{comment}

Particle physicists seek to understand the building blocks of the universe and how they interact.
An understated search to characterize the fundamental constituents of nature which can be built up into the world we see.
In this quest there has been no better tool than the synchrotron, a circular accelerator which collides particles at speeds approaching that of light.
As the accelerators reach higher and higher energies, physicists are able to probe smaller distance scales and even create heavy, short lived particles which are otherwise inaccessible.
The standard model (SM) of particle physics is the codification of over a century of study.
It describes all of the observed elementary particles, their properties, and the electromagnetic, weak, and strong forces through which they interact.
The standard model, a specific framework born out of quantum field theory (QFT), has predicted quantities and been proven accurate time and time again.
Yet until recently it remained an incomplete model, at least experimentally.

One of the primary missions of the Large Hadron Collider (LHC), the worlds highest energy particle accelerator located at the European Organization for Nuclear Research (CERN), was to search for the last remaining particle in the SM.
On July 4th, 2012 the ATLAS (A Toroidal LHC Apparatus) and CMS (Compact Muon Solenoid) collaborations at the LHC simultaneously confirmed the discovery of a new boson~\cite{20121,201230}.
Since its discovery, the particle has been shown to be consistent with the long proposed the Higgs boson, said to give mass to itself and all of the other massive particles through the process of electroweak symmetry breaking.
It took almost 50 years for experimentalists to confirm the existence of the boson first proposed in 1964 as the spin zero mediator to the standard models only scalar field.

Using 19.7\fbinv of 8\tev data from the CMS experiment at CERN, the Higgs boson mass was measured to be \longmass{125.7}{0.3}{0.3}{\gev}\footnote{Unless otherwise indicated this document will use natural units, where $c=\hbar=1$.}\footnote{This measurement has subsequently been improved by combining the ATLAS and CMS measurements. The measured Higgs mass as of 2015 was \longmass{125.09}{0.21}{0.11}{\gev}~\cite{Aad:2015zhl}.} by five major decay modes: $H\rightarrow\gamma\gamma$, $H\rightarrow\tau\tau$, $H\rightarrow{bb}$, $H\rightarrow{ZZ}$, and $H\rightarrow{WW}$~\cite{CMS-PAS-HIG-13-005}.
Since then, the experiment has entered a phase of intense study of the new particle.
Every property of the new boson and all of its decay channels must be studies in great detail to confirm that it is indeed the SM Higgs boson and not a very similar particle.
Currently the properties of the new boson are consistent with those predicted by the SM, but any deviation from the SM predictions could point to some new, as yet unexplored physics.

This dissertation will present a search for the 125\gev Higgs boson in the the\newline\HWWlnujj decay channel using 8\tev proton-proton data collected by the CMS detector.
Although the \HWWlnujj channel was used in the original combined limit, the previous search was not sensitive to the ``low mass'' Higgs, but only to 
$\MH>2\MW$~\cite{CMS-PAS-HIG-13-027}.\footnote{The lowest search mass was $\MH=170\gev$.}
Because the Higgs mass is less than two times the mass of the W boson, at least one of the W bosons must be created ``off-shell'', meaning that its measured mass is not $\sim80\gev$.
On top of that, the presence of a neutrino makes it a challenge to fully reconstruct the initiating particle.
For these reasons the $WW\rightarrow{l\nu}{l\nu}$ decay channel was the most sensitive of the $WW$ channels during the 2012 combination.
Nevertheless this analysis will search for the low mass Higgs boson in the semi-leptonic channel using a matrix element (ME) technique to boost the signal extraction sensitivity.

This dissertation will be organized in the following way.
Chapter~\ref{ch:theoretical_framework} will present an overview of the standard model, the Higgs mechanism, and a brief introduction to how the Higgs can point to physics beyond the standard model (BSM).
The LHC and CMS will be described in chapter~\ref{ch:LHC_CMS}.
Chapter~\ref{ch:event_reconstruction} describes the reconstruction of an event at CMS and all of the final physics objects.
Chapter~\ref{ch:analysis} discusses the analysis work-flow from data samples used to signal extraction techniques while the results are presented in chapter~\ref{ch:results}.
Chapter~\ref{ch:conclusion} gives some concluding remarks.

