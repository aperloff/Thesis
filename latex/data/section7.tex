%!TEX root = ../TAMUTemplate.tex
%%%%%%%%%%%%%%%%%%%%%%%%%%%%%%%%%%%%%%%%%%%%%%%%%%%
%
%  New template code for TAMU Theses and Dissertations starting Fall 2016.
%
%  Author: Sean Zachary Roberson
%	 Version 3.16.09
%  Last updated 9/12/2016
%
%%%%%%%%%%%%%%%%%%%%%%%%%%%%%%%%%%%%%%%%%%%%%%%%%%%
%%%%%%%%%%%%%%%%%%%%%%%%%%%%%%%%%%%%%%%%%%%%%%%%%%%%%%%%%%%%%%%%%%%%%%
%%                           SECTION VII
%%%%%%%%%%%%%%%%%%%%%%%%%%%%%%%%%%%%%%%%%%%%%%%%%%%%%%%%%%%%%%%%%%%%%



\chapter{\texorpdfstring{\uppercase {Conclusions}}{Conclusions}}
\label{ch:conclusion}

This dissertation has presented a search for the 125\gev Standard Model Higgs boson in th the \HWWlvjj decay channel.
The search used 19.7\fbinv of 8\tev proton-proton collision data from the CMS experiment collected during the 2012 run of the LHC.
The background predictions used in the analysis were derived from both simulation and data-driven techniques and a significant amount of time was put into validating the background modeling.
The event selection was chosen based on the signal kinematics, but kept relatively loose to ensure enough of a training ensemble for an analysis using a boosted decision tree (BDT).
Besides the BDT based discriminant, a matrix element technique was used to increase the sensitivity of the analysis.
No direct observation of the Standard Model Higgs boson can be made at this time, though limits on its production cross section have been made at the 95\% confidence level using a modified frequentist approach.
A limit of 13.9 times the standard model cross section was set after combining all lepton and jet categories.
This limit is the first to be set in the \HWWlvjj channel for a Higgs mass of \joinsym{\MH}{=}{125\gev} at either the CMS or ATLAS experiments.

Now that Run2 of the LHC has collected significantly more data and at a center of mass energy of \CM{13\tev}, it is possible to perform a similar analysis to this one with improved results.
This increase in energy corresponds to an increase in gluon-gluon fusion Higgs production of approximately 2.4 times~\cite{Baglio:2015wcg}, while the \Wjets production cross section will only increase by approximately 1.7 times~\cite{SMCrossSectionsat13TeV}.
%ggH production cross section: 46.68/19.27=2.4224182667
%WJets production cross section: (20508.9*3)/(12234*3)=1.6763854831
This will lead to a higher signal fraction, which should be visible given improvements in background modeling and reconstruction techniques.
Even now there have been advances in high performance computing which will reduce the time to perform a matrix element analysis by orders of magnitude~\cite{1742-6596-664-9-092009}.
Additionally, advances in machine learning will significantly speed up analyses relying on Monte Carlo integration techniques~\cite{Bendavid:2017zhk}.

Although it may seem that the Run1 discovery of the Standard Model Higgs boson was the final say on the matter, I am optimistic that even more stringent measurements of this Higgs decay channel can be made in the coming years.