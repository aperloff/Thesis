%!TEX root = ../TAMUTemplate.tex
%%%%%%%%%%%%%%%%%%%%%%%%%%%%%%%%%%%%%%%%%%%%%%%%%%%
%
%  New template code for TAMU Theses and Dissertations starting Fall 2016.
%
%  Author: Sean Zachary Roberson
%	 Version 3.16.09
%  Last updated 9/12/2016
%
%%%%%%%%%%%%%%%%%%%%%%%%%%%%%%%%%%%%%%%%%%%%%%%%%%%
%%%%%%%%%%%%%%%%%%%%%%%%%%%%%%%%%%%%%%%%%%%%%%%%%%%%%%%%%%%%%%%%%%%%%%
%%                           SECTION VII
%%%%%%%%%%%%%%%%%%%%%%%%%%%%%%%%%%%%%%%%%%%%%%%%%%%%%%%%%%%%%%%%%%%%%



\chapter{\texorpdfstring{\uppercase {Conclusions}}{Conclusions}}
\label{ch:conclusion}

\begin{comment}
Joey:

Using the entire 19.1 fb−1 of data collected at 8TeV no direct observation of the Higgs was seen
2520 in the H → WW → lνjj decay channel. Due to the large amount of background, while Higgs 2521 events certainly exist in our data, we do not achieve the sensitivity needed to discriminate it 2522 from our backgrounds. Thus, in the absence of a significant excess of events in data indicative of
2523 our signal, we can set upper limits on the production rate of H → WW → lνjj. Two methods 2524 of setting limits were employed using the information from our trained and optimized Boosted 2525 Decision Tree (BDT)s. By placing a cut on the BDT discriminant output we were able to set 2526 an upper limit on the production cross section of 16.42, using the statistical methods described 2527 above. From simulations alone the expected factor was 13.91, a difference of less than 1-σ from 2528 the observed value.
2529
Using the full BDT output shape we were also able to set limits on the production cross 2530 section. As noted above, a large uncertainty in the shape between data and simulation produced
2531 some curious results. By using the combined information in the 2, 3, and ≥ 4 jet bin shapes, 2532 this uncertainty was better constrained. Using this method an upper limit of 8.86 times the 2533 production cross section is measured, which falls within 2-sigma of the expected value of 4.98 2534 seen from simulations alone.






Joey:

The results for a search for the Higgs Boson in the H→WW→lνjj in pp collisions at √s = 8 TeV 2538 center of mass energy have been presented. This analysis begins with the production of protons in
2539 the LHC accelerator complex, traveling through many complex systems on their way to a collision 2540 at √s = 8 TeV at the center of the CMS detector. The superior tracking and reconstruction of 2541 particles in CMS led to over 19 fb−1 of data collected in 2012 that was used in this analysis.
2542
Once collected, a search was performed for our signal in a final state that included one isolated
2543
lepton, one neutrino (indicated by E
/T ), and two jets. We further required that the jets not be
2544 b-tagged, restricting our sample to light flavor jets that are more common from a W decay. 2545 The search region was divided into categories based on the number of jets in the event, using
2546 categories of 2, 3, or ≥4 jets. For each category we trained a Boosted Decision Tree (BDT) by 2547 using kinematic variables as inputs, with each category optimized for maximum signal extraction 2548 potential. 2549 We looked at two methods for signal extraction, a counting experiment that took advantage 2550 of the BDT by using it as a superior discrimination variable to cut away background, and as 2551 a shape analysis using the entire BDT output shape to separate signal from background. No 2552 significant excess was seen seen using either method, so an upper limit on the production cross 2553 section was placed. Using a counting experiment we set a limit of 16.4 times the standard model, 2554 and using shape based signal extraction we were able to lower this limit to 8.86.

Though this analysis did not have the sensitivity to observe the Higgs directly, I am optimistic
2556 that in the future the increase of data will make this possible. With Run II at the LHC just 2557 beginning, and an increase in the Higgs production cross section at √s = 13 TeV, there will 2558 definitely be more signal out there to find. The increase in luminosity and pileup will require new 2559 and unique ways to reduce the backgrounds seen in this channel, but though careful background 2560 modeling I believe it’s possible.
142
143 2561 Finally, combining this analysis with others looking for the same final state could increase the
2562 sensitivity. Use of Matrix Element values for particle production could serve as a good addition 2563 to the kinematic information of the event, producing a result more sensitive to probing the limits 2564 of the Standard Model.
\end{comment}


\section{Updates For Future Searches}